\section{Feasibility Analysis}
The project is special in this way that its complexity can be exponentially increased with every new language, library, framework, or database that we choose to support. Hence, the scope of the project is very important in defining the feasibility of this project.

\subsection{Technical Feasibility}
Tools such as Create-React-App (CRA), and Create-Next-App \todo{Change for some other tool} (both of which are pretty popular) already exist which provide a boilerplate code for their respective technologies. For example Create-React-App sets up React library with Babel as compiler and Webpack as asset bundler. From a perspective of software architecture, building a tool which sets up a database along with a design library (e.g. Tailwind) with React is very similar, but a bit more difficult to implement. Hence, it is technically practical to build such a project.

One of our members (Muhammad Saad) has worked on similar projects on a smaller scale; his experience will be valuable in developing this project. Other members of the team have worked on the main language that we are focusing on (i.e. Javascript) and on many of its packages, so that will help make setting up boilerplate code.

\subsection{Operational Feasibility}
This is a tool from developers for developers. People who are already familiar with the technologies but want to skip the part where one scours through documentation on the internet, installs packages, and sets up files, environment, etc.

Firstly, We see this as being useful to freelance developers, and agencies, which take on greenfield projects of similar types from multiple clients. Instead of wasting time on setting up the project, the developer/company can save time by cutting to the chase and working on what their client actually wants.

Secondly, startups who have limited resources and want to beat their competitors to the market by rapid prototyping and development can use this tool to quickly set up a base on which they can develop only those features that make their product/service special.

\subsection{Economical Feasibility}
There are the following costs involved in this project:
\begin{enumerate}
  \item Time spent in research, development, and documentation of the project
  \item Cost of making and hosting a website for documentation
  \item If we implement this project as a online service then we also need to include the cost of running a server (or cloud computer instance) which can generate boilerplate for our users
\end{enumerate}
Our team is not doing this project for profit. The project's code and documentation will be open sourced and hence will be freely available to anyone who wants to use it. Therefore for an end user, the benefits of being able to set up a project in minimal time without any added cost will definitely be worthwhile.

\subsection{Time Feasibility}
The timeline of your project will go here. (Leave this section blank).
