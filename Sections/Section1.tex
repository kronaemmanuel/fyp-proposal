\section{Introduction}

\subsection{Terminology}

\textbf{Boilerplate Code:} For the purposes of our project, the term boilerplate code means the initial code that a developer needs to set up to get a project of some specific stack working. It is the 'scaffolding' code with basic features that can then be used as a base to start a project.

\textbf{Configuration:} A project's configuration refers to the different languages, frameworks, and libraries being used in a project. This configuration is needed by our tool to generate boilerplate code for the user.

\subsection{Problem Statement and motivation}

We initially faced this problem ourselves as freelance developers working for multiple clients when we had to go through very similar steps for each client's app to reach a point after which we could incorporate stuff that was specific to the client's demands. These repeated configurations include but are not limited to setting up frontend boilerplate, setting up backend boilerplate, connecting the frontend with the backend via a REST API, connecting a database inside the project. All of these mentioned configurations have the same code across multiple projects. Yet, we had to spend time on it again and again.

We also presented this idea to a web and mobile app development agency who confirmed that they could use this tool as an optimization in their own workflow. They were previously using hand-written internal procedures to make it easier for the developers to set up a new project whenever they start a new project for a client. However, this tool could optimize that even further by cutting the time required to set up a project from several hours to a few seconds.

The motivation behind proposing this idea is to automate all this grunt work. And build a tool that can do all these configurations within minutes, if not seconds.

\subsection{Overview of the Project}

Our software generates a boilerplate code repository based on the \emph{configuration} that a user wants. A \emph{configuration} can consist of the libraries, frameworks, databases, features, and other technologies that the user wants to add to their project. As an example, a user might want to setup a \emph{ReactJS} frontend with a \emph{Express} backend along with \emph{TailwindCSS} as their design library and \emph{Firebase} as their database. They can select all these options through the software's interface and generate boilerplate code according to their own needs.

We plan to build one of the following types of interfaces to interact with our software.
\begin{enumerate}
  \item \textbf{Command Line Tool:} We can develop write a Command Line Interface (CLI) with Node.js (a Javascript runtime) and publish it on Node Package Manager (NPM) repository. The user can select the configuration they want either by passing arguments to the command line, or by writing them in a configuration file.
  \item \textbf{Website:} We can develop a website where the user can enter configuration by clicking through and checking/unchecking various options. They can then either download the generated boilerplate code or have it setup as a git repository.
\end{enumerate}

\subsection{Prior work}

When Facebook introduced React.js, the software developers had to do all the configurations themselves to get started with building a React.js project. Facebook acknowledged this issue, and developed create-react-app (CRA). CRA is a tool that provides you a boilerplate code that you can use to get started with React.js.  create-react-app uses webpack for bundler, babel to transpile ES6 JavaScript to older version and other tools to provide you a clean boilerplate without the need for you to configure everything on your own. Now, create-react-app has more than 100,000 downloads weekly and is widely being used by the developer community.

\subsection{Our Contribution}

There are a few other tools like create-react-app that generates boilerplates. All of these tools are technology specific. That means they only create boilerplate for their particular technology. create-react-app creates a React.js boilerplate. create-next-app creates a Next.js boilerplate. There is no generic tool in the market at the moment. Yet the concept exists of building automation tools to generate boilerplate.

We are proposing to create such a generic tool that not only generates React.js frontend but also creates Node.js backend’s boilerplate at the same time. A tool that also produces Python frameworks’ boilerplates. A tool that you can also integrate services like firebase.

