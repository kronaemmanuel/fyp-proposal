\section{Target Audience}
Our target audience is developers, who are already familiar with the technology stack that they are generating boilerplate code for, and know what to do after a project is setup for them.

\subsection{Freelance Developers}
Our team consists of developers who have done freelance work, and we initially noticed the need for this tool in our own lives. Since we were working on similar projects for different client, we often went through the same steps again and again to set up projects for different clients. Our tool would automate all this and allow us to get to our finished product faster.

\subsection{Startups}
Startups are often strapped for cash and time. If they want to beat their competitors to the market, then they wouldn't want to waste time on setting up basic amenities like having a database connected or having user authentication working. We also have to consider that doing this all by hand blocks a lot of tasks, since they are waiting for components to be connected before work can be started on them. Instead, startups can save a lot of working hours, by using our tool and then focusing on the stuff that makes their own product special.

\subsection{Companies}
Companies/organizations usually develop projects using the same type of technologies(frameworks & libraries) or follow the same paradigm. One of the companies we interviewed, revealed that since they always worked on Django (a Python web framework) projects with different variations, so they shared a lot of code between the codebases for different clients. They had written down procedures of how a developer can start a new project in the same way that their other projects were set up; the procedure included a lot of copying and pasting code from previous projects. Our tool can eliminate the need to do all this by hand and will also save working hours of developers spent in following such procedures.

